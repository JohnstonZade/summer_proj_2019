\documentclass[12pt,letterpaper]{article}
\usepackage{fullpage}
\usepackage[top=2cm, bottom=4.5cm, left=2.5cm, right=2.5cm]{geometry}
\usepackage{amsmath,amsthm,amsfonts,amssymb,amscd}
\usepackage{lastpage}
\usepackage{enumerate}
\usepackage{fancyhdr}
\usepackage{mathrsfs}
\usepackage{xcolor}
\usepackage{graphicx}
\usepackage{listings}
\usepackage{hyperref}
\usepackage[nodayofweek, level]{datetime}
\usepackage{bbm}
\usepackage{cases}
\usepackage{esint}

\setlength{\parindent}{0.0in}
\setlength{\parskip}{0.05in}

\usepackage{subcaption}
\usepackage{bold-extra}
\usepackage{epigraph}

\hypersetup{
    colorlinks=true,
    linkcolor=blue,
    filecolor=magenta,
    urlcolor=blue,
    pdftitle={Sharelatex Example},
    pdfpagemode=FullScreen,
    }

%% ----- COMMANDS -----
% Headings
\newcommand\course{Project Writeup}
\newcommand{\theorem}[1]{\underline{\textbf{#1}} \\}
\newcommand{\mybox}[1]{\noindent\fbox{\parbox{\textwidth}{#1}}}

% Notation
\newcommand{\set}[1]{\{#1\}}
\newcommand{\C}{\mathbb{C}}
\newcommand{\N}{\mathbb{N}}
\newcommand{\Q}{\mathbb{Q}}
\newcommand{\R}{\mathbb{R}}
\newcommand{\Z}{\mathbb{Z}}
\newcommand{\BXi}{\boldsymbol{\xi}}
\newcommand{\BNu}{\boldsymbol{\nu}}
\newcommand{\BX}{\boldsymbol{x}}
\newcommand{\BU}{\boldsymbol{u}}
\newcommand{\inv}{^{-1}}
\newcommand{\modulo}[1]{\text{ mod }#1}
\newcommand{\kernel}[1]{\text{ker }#1}
\newcommand{\LeftImplies}{\underline{$\Leftarrow$:} }
\newcommand{\RightImplies}{\underline{$\Rightarrow$:} }
\newcommand{\cycle}[1]{\langle #1 \rangle}

\pagestyle{fancyplain}
\headheight 35pt
\rhead{\course}
\lfoot{}
\cfoot{}
\headsep 1.5em

\fancypagestyle{firststyle}
{
  \fancyhf{}
  \headheight 35pt
  \chead{\textbf{\Large Turbulence in Plasma}} % Edit for title
  \rhead{\course \\ Zade Johnston}
  \lfoot{}
  \cfoot{}
  \headsep 1.5em
}

\graphicspath{{./images/}}

\begin{document}

  \thispagestyle{firststyle}
  \section*{Week 1: 11 Nov - 13 Nov}

  Started this week learning about how to use MPI in Athena++. Followed the \href{https://github.com/PrincetonUniversity/athena-public-version/wiki/Running-3D-MHD-with-OpenMP-and-MPI}{3D blast wave tutorial} and managed to get it working on Thunderbird with HDF5. Jono also recommended learning a text editor, so I spent an hour learning the basics of Vim. \textbf{Note:} HDF5 is better to use than .vtk with parallel computing as it allows all the processors being used to write the data in one file compared to a file for each processor that need to be joined (less hassle).

  Jono then gave me an Athena++ hydrodynamic turbulence input script to play around with, and some MATLAB scripts that analyse the energy spectrum of the fluid in question. At the moment, we model the fluid in a cube. There are two different modes that we're wanting to focus on: decaying turbulence (disturb the fluid initially then leave it to its own devices) and continuously driven turbulence. Ran the Athena++ code with 3 different grid sizes for both modes; see screenshots below.

  \begin{figure}[!h]
   \centering
  \begin{subfigure}{0.3\textwidth}
  \includegraphics[width=0.9\linewidth, height=0.9\linewidth]{Wk1/32_forced_rho.png}
  \caption{Grid Size: 32}
  \label{fig:32rho}
  \end{subfigure}
  \begin{subfigure}{0.3\textwidth}
  \includegraphics[width=0.9\linewidth, height=0.9\linewidth]{Wk1/64_forced_rho.png}
  \caption{Grid Size: 64}
  \label{fig:64rho}
  \end{subfigure}
  \begin{subfigure}{0.3\textwidth}
  \includegraphics[width=0.9\linewidth, height=0.9\linewidth]{Wk1/128_forced_rho.png}
  \caption{Grid Size: 128}
  \label{fig:128rho}
  \end{subfigure}

  \caption{Face-on view of the 3D forced turbulence simulations with different grid sizes; density plotted}
  \label{fig:forcedturb}
  \end{figure}

  These were just the simulations with none of the parameters changed in the input script. Using the MATLAB scripts I also obtained the energy spectrum of the fluid; was very rough due to the low resolution. Wanted to try start a $256^3$ grid size simulation before I left on Wednesday but didn't have enough time to set up; will try again later. Thursday and Friday of this week were spent at the \href{https://otagocarpentries.github.io/2019-11-14-otago/}{Otago Software Carpentry Workshop}.

  \textbf{Next Week:} Play around with parameters. Want to run the larger grid size simulations to obtain a better energy spectrum that fits the $k^{-5/3}$ law, will add screenshots of spectrum then. Plot the total energy over time for both modes; should observe fluctuations in the energy for forced turbulence.


  \section*{Week 2: 18 Nov - 22 Nov}
  This week I ran bigger simulations for both decay and forced turbulence from last week in order to be able to plot the energy spectrum and time evolution. The grid size ranged from 32 to 256, and runs over 30 seconds. All simulations left the parameters in \verb|athinput.turb| unchanged.

  Calculated the turnover time $\tau \sim L/u_l$ of eddies on the scale of the box to get an idea of the timescales involved in the energy cascade. This is important in the decaying case as all the input energy dissipates within a few turnover times, so this allowed me to get an approximate time range to average the energy spectrum over. For these simulations, I used $L=1$ and $u_L = \sqrt{u^2_x+u^2_y+u^2_z}$ taken at the start of the simulation from the \verb|Turb.hst| file.

  \textbf{Continuous Forcing:} For the continously forced case, the 256 grid size gave the best result. This is expected as it is able to simulate smaller scale eddies compared to lower resolution simulations, allowing more of the energy cascade to be observed. We see that the energy spectrum does follow the $k^{-5/3}$ law (shown in Fig. \ref{fig:256contspec}) for a given range of wavevectors.

  The energy evolution (Fig. \ref{fig:256conteng}) shows an increase in kinetic energy until it levels out after a few turnover times. This leveling out is due to the energy dissipation rate matching the energy input rate from the forcing. There is still some variation around the mean value, which arises from fluctuations due to turbulence.

  \begin{figure}[!h]
   \centering
  \begin{subfigure}{.5\textwidth}
    \centering
  \includegraphics[width=0.8\linewidth, height=0.9\linewidth]{Wk2/cont_turb_256_default.png}
  \caption{Averaged Spectrum}
  \label{fig:256contspec}
  \end{subfigure}
  \begin{subfigure}{.5\textwidth}
    \centering
  \includegraphics[width=0.8\linewidth, height=0.9\linewidth]{Wk2/cont_turb_256_default_energy.png}
  \caption{Energy Evolution}
  \label{fig:256conteng}
  \end{subfigure}

  \caption{Plots showing the energy spectrum and time evolution of the 256 grid size continuous forcing simulation}
  \label{fig:256contenergy}
  \end{figure}

  \textbf{Decay:} The energy spectrum (Fig \ref{fig:128decayspec}), averaged over the first two turnover cycles, is not as well defined as the continuous case. I think this is because the energy depends on time instead of being approximately constant. The energy evolution (Fig. \ref{fig:128decayeng}) shows that the total kinetic energy decreases over time due to dissipation from viscosity.

  For the spectrum evolution (Figure \ref{fig:128decayevo}) I took snapshots of the spectrum at 3 second intervals, with a snapshot at 0.1 seconds to observe the energy input (the curve sharply peaked at $k \sim 10$). We see that the energy cascades down through the length scales directly after input (at 6 seconds), and then decreases as it is dissipated as heat at the micro scale (shown by the ``sinking" of the spectrum). This is expected as we saw that the total energy in Fig. \ref{fig:128decayeng} is decreasing.

  \begin{figure}[h]
   \centering
  \begin{subfigure}{0.5\textwidth}
  \centering
  \includegraphics[width=0.8\linewidth, height=0.8\linewidth]{Wk2/decay_turb_128_default.png}
  \caption{Averaged Spectrum}
  \label{fig:128decayspec}
  \end{subfigure}
  \begin{subfigure}{0.5\textwidth}
  \centering
  \includegraphics[width=0.8\linewidth, height=0.8\linewidth]{Wk2/decay_turb_128_default_energy.png}
  \caption{Energy Evolution}
  \label{fig:128decayeng}
  \end{subfigure}
  \begin{subfigure}{0.5\textwidth}
  \centering
  \includegraphics[width=0.8\linewidth, height=0.8\linewidth]{Wk2/decay_turb_128_default_evo.png}
  \caption{Spectrum Evolution}
  \label{fig:128decayevo}
  \end{subfigure}

  \caption{Plots showing the energy spectrum and time evolution of the 128 grid size decay simulation}
  \label{fig:128decayenergy}
  \end{figure}

  The magnetic and thermal energy had no relevance to these simulations; it's just part of the MATLAB script that I forgot to remove.

  I learnt that it's good to run the simulations at different resolutions starting with the lowest as it allows you to get a rough idea of what is going to happen without the expense of computing time. It also helps as you can compare with the model to see whether the configuration used is worth investigating. The higher resolutions could differ as turbulence depends strongly on all length scales in the inertial subrange, which are included in the bigger simulations, but it still helps to see what could happen.


\end{document}
