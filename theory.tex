\documentclass[12pt,letterpaper]{article}
\usepackage{fullpage}
\usepackage[top=2cm, bottom=4.5cm, left=2.5cm, right=2.5cm]{geometry}
\usepackage{amsmath,amsthm,amsfonts,amssymb,amscd}
\usepackage{lastpage}
\usepackage{enumerate}
\usepackage{fancyhdr}
\usepackage{mathrsfs}
\usepackage{xcolor}
\usepackage{graphicx}
\usepackage{listings}
\usepackage{hyperref}
\usepackage[nodayofweek, level]{datetime}
\usepackage{bbm}
\usepackage{cases}

\setlength{\parindent}{0.0in}
\setlength{\parskip}{0.05in}


%% ----- COMMANDS -----
% Headings
\newcommand\course{Project Writeup} % Edit for course name
\newcommand{\theorem}[1]{\underline{\textbf{#1}} \\}

% Notation
\newcommand{\set}[1]{\{#1\}}
\newcommand{\C}{\mathbb{C}}
\newcommand{\N}{\mathbb{N}}
\newcommand{\Q}{\mathbb{Q}}
\newcommand{\R}{\mathbb{R}}
\newcommand{\Z}{\mathbb{Z}}
\newcommand{\inv}{^{-1}}
\newcommand{\modulo}[1]{\text{ mod }#1}
\newcommand{\kernel}[1]{\text{ker }#1}
\newcommand{\LeftImplies}{\underline{$\Leftarrow$:} }
\newcommand{\RightImplies}{\underline{$\Rightarrow$:} }
\newcommand{\cycle}[1]{\langle #1 \rangle}
\newcommand{\B}[1]{\mathbf{#1}}

\pagestyle{fancyplain}
\headheight 35pt
\rhead{\course}
\lfoot{}
\cfoot{}
\headsep 1.5em

\fancypagestyle{firststyle}
{
  \fancyhf{}
  \headheight 35pt
  \chead{\textbf{\Large Theory}} % Edit for title
  \rhead{\course \\ Zade Johnston}
  \lfoot{}
  \cfoot{}
  \rfoot{\small\thepage}
  \headsep 1.5em
}

\begin{document}
  \section*{Theory}
  \textbf{Kolmogorov energy cascade law: } Energy that is put into a turbulent fluid at the largest relevant scale doesn't immediately dissipate as heat. Instead it is cascaded down the length scales of the fluid (from macro to micro) in the formation of increasingly smaller eddies, and is dissipated as heat at the microscopic scale due to viscous effects. This produces a kind of order out of the (seemingly) chaos of turbulence.

  For a given eddy length scale $l$, we can find a relation between the wavenumber ($k=2\pi/l$) and the energy contained in the fluid $E(k)$. This theory was developed by Kolmogorov and says $E(k)\sim k^{-5/3}$ for a given range of wavenumbers, called the \textbf{inertial subrange}. This law fails at small scales (large $k$) for the reason explained above, and at large scales (small $k$) as this is where the energy is being injected into the system.

  \textbf{Derivation using dimensional analysis:} The Navier-Stokes equation is
  $$
    \frac{\partial \B{u}}{\partial t} + (\B{u}\cdot\nabla)\B{u} = -\nabla p + \nu \nabla^2\B{u}
  $$
  Turbulence arises from the nonlinear term $(\B{u}\cdot\nabla)\B{u}$ (represents the energy cascade), and energy is dissipated via the viscosity term $\nu \nabla^2\B{u}$. Using dimensional analysis, at a given scale $l$ with bulk velocity $u_l$ we have $|(\B{u}\cdot \nabla)\B{u}|\sim u^2_l/l$ and $|\nu \nabla^2 \B{u}|\sim \nu u_l / l^2$ (using $\nabla \sim 1/l$).

  Reynolds number Re is the ratio of the turbulence term to the viscosity term:
  $$
    \text{Re} \sim \frac{|(\B{u}\cdot \nabla)\B{u}|}{|\nu \nabla^2 \B{u}|} = \frac{u_l l}{\nu}
  $$

  The viscous term is much smaller than the nonlinear term for large $l$ (due to the factor of $1/l^2$). This means that at large scales energy can't dissipate as heat, so the nonlinear term effectively cascades energy through eddies as it has nowhere else to go. Only when the two terms are comparable (Re $\sim 1$) can the energy be dissipated as heat. This also explains the self-similarity in the energy spectrum across the inertial subrange.

  For a given length scale $l$ and velocity $u_l$ (this is the \textbf{structure function}), the energy is $E\sim u^2_l$. The turn-over time for an eddy is $\tau\sim l/u_l$. The rate of energy dissipation is then $E/\tau \sim u^3_l/l=\text{const.}$ across the inertial subrange due to self-similarity.

  We then have $u_l\sim l^{1/3}\sim k^{-1/3} \equiv u_k$. The energy in all eddies with wavenumber $\geq k$  is given by
  $$
  E = u^2_k \sim k^{-2/3} \sim \int_k^\infty E(\kappa) \ d\kappa  \implies E(k) \sim k^{-5/3}
  $$
  giving us Kolmogorov's result.

  The structure function $\langle[u_l]^2 \rangle = \langle [u(x+l)-u(x)]^2\rangle$ measures how smooth the velocity distribution is for a given length scale. It is small at large $l$ as velocity is continous at that scale, and large at small $l$ as velocity can change appreciatively over that scale.

  \textbf{MHD Equations: }
  Using $D/Dt = \partial / \partial t + \B{u}\cdot\nabla$ as the Lagrangian derivative.
  % \begin{align*}
  %   \frac{D\rho}{Dt} &= -\rho \nabla\cdot\B{u} \\
  %
  %   \rho \frac{D\B{u}}{Dt} &= -\nabla\left(c_s^2\rho + \frac{B^2}{8\pi}\right) + \frac{\B{B}\cdot\nabla\B{B}}{4\pi} + \nu \nabla^2 \B{u}\\
  %
  %   \frac{D\B{B}}{Dt} = \B{B}cdot\nabla\B{u} - \B{B}\nabla\cdot\B{u} + \eta \nabla^2\B{B}
  % \end{align*}

  \begin{align}
        \frac{D\rho}{Dt} &= -\rho \nabla\cdot\B{u} \\
        \rho \frac{D\B{u}}{Dt} &= -\nabla\left(c_s^2\rho + \frac{B^2}{8\pi}\right) + \frac{(\B{B}\cdot\nabla)\B{B}}{4\pi} + \nu \nabla^2 \B{u} \\
        \frac{D\B{B}}{Dt} &= (\B{B}\cdot\nabla)\B{u} - (\nabla\cdot\B{u})\B{B} + \eta \nabla^2\B{B}
  \end{align}
  Equation 1 is the mass continuity equation (what comes in must go out).

  Equation 2 gives the time evolution of the fluid due to forces, and is obtained from the Navier-Stokes equation:
  \begin{itemize}
    \item $-\nabla\left(c_s^2\rho + \frac{B^2}{8\pi}\right)$: Says that the higher the mass and magnetic field density, the greater the force is on the fluid in a direction away from that density.

    \item $\frac{\B{B}\cdot\nabla\B{B}}{4\pi}$: acts as a ``tension" on the magnetic field lines; i.e. if the magnetic field lines are perturbed it acts to restore to its inital condition.

    \item $\nu \nabla^2 \B{u}$: usual viscosity term in the fluid.
  \end{itemize}

  Equation 3 determines the evolution of the magnetic field. Looks similar to the mass continuity equation, and is obtained from manipulating the Lorentz force equation and Maxwell's equations. Often called the ``frozen in" condition as field lines move with the fluid:
  \begin{itemize}
    \item $(\B{B}\cdot\nabla)\B{u}$ and $(\nabla\cdot \B{u})\B{B}$: acts to stretch the field lines with the fluid. The first is stretching due to shear flows while the second is stretching due to compressible flows (not as important as we usually have $\nabla\cdot\B{u} = 0$).

    \item $\eta \nabla^2\B{B}$: The ``viscosity" term of the magnetic field; $\eta$ is the resistance of the charged fluid. Gives a similar spectrum for the energy in the magnetic field as the one for the kinetic energy.
  \end{itemize}

  \textbf{Critical Balance:} For turbulence in MHD, critical balance is the assumption that
  $$
    k_\perp u_{\perp k} \sim k_\| v_A
  $$
  across all scales in the inertial range, where $k_\perp$ and $k_\|$ are the components of $\B{k}$ perpendicular and parallel to the magnetic field. Unlike in hydrodynamic turbulence, eddies in MHD spread out along the magnetic field lines as energy is cascaded down.

  Alf\'ven waves are shear waves (i.e. doesn't compress fluid $\iff \nabla\cdot\B{u}=0$) that travel through a charged fluid along magnetic field lines. Alf\'ven waves have the following dispersion relationship:
  $$
    \omega = k_\| v_A
  $$
  where $k_\|$ is the component of the wavevector parallel to the magnetic field and $v_A = B_0 / \sqrt{4\pi \rho}$ is the Alf\'ven velocity at which the waves move, where $B_0$ and $\rho$ are the magnetic field strength and fluid density. Because they travel along the magnetic field lines, viscosity doesn't affect them as much as it does sonic waves because the particles can't ``talk" to each other. This means that smaller scale structures that would otherwise have been lost due to viscosity survive; this is what allowed finer details in the solar wind to be observed near Earth.

  When simulating turbulence in MHD we choose a box whose size is small compared to the length scales of the forcing. We can consider the ``forcing" on this box to be the energy coming from the energy cascade at larger scales. We align the box such that the longest edge is parallel to the magnetic field lines. Call this length $L_x$, and $L_y$ the length of the side perpendicular to the magnetic fields. Then:
  $$
    k_\| = \frac{2\pi}{L_x}, \ k_\perp = \frac{2\pi}{L_y}
  $$
  In order to respect the assumption of critical balance, these lengths must satisfy
  $L_y/L_x \sim u_{\perp k}/v_A$.

\end{document}
